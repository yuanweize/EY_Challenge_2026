\section{Task Overview}

\subsection{Goal}
The primary objective of the ``2026 Optimizing Clean Water Supply'' challenge is to develop machine learning models capable of predicting water quality parameters in various river locations across South Africa. This is critical for ensuring sustainable water resource management.

\subsection{Target Parameters}
Participants are required to predict the following key water quality indicators:
\begin{itemize}
    \item \textbf{Total Alkalinity}: A measure of the water's ability to neutralize acids.
    \item \textbf{Electrical Conductance}: An indicator of the concentration of dissolved salts.
    \item \textbf{Dissolved Reactive Phosphorus}: A critical nutrient often associated with pollution and eutrophication.
\end{itemize}

\subsection{Data Sources}
The challenge leverages a combination of remote sensing and ground-level data:
\begin{itemize}
    \item \textbf{Satellite Imagery}: Landsat Level-2 data for spectral analysis of water bodies.
    \item \textbf{Weather Data}: TerraClimate datasets providing climatological context.
    \item \textbf{Ground Truth}: Historical environmental measurements from 2011 to 2015.
\end{itemize}

\subsection{Timeline}
\begin{description}
    \item[Enrollment:] Opens January 20, 2026.
    \item[Evaluation Start:] March 14, 2026.
    \item[Challenge End:] May 6, 2026.
\end{description}

\subsection{Tools}
Participants are encouraged to use the Snowflake platform for model development, due to the volume and nature of the data involved. Notebooks and setup scripts are provided for both Snowflake-specific and general Jupyter environments.
