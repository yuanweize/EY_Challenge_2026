\section{Eligibility / 参赛资格}
The challenge is open to / 挑战赛面向以下人群开放:
\begin{itemize}
    \item \textbf{University Students / 在校大学生}: Currently enrolled in an accredited institution. (目前在正规院校就读)
    \item \textbf{Early Career Professionals / 早期职业专业人士}: Individuals with less than 5 years of professional experience. (拥有少于5年专业经验的个人)
\end{itemize}

\begin{infobox}
Participants who do not meet these criteria may still join the challenge but are \textbf{not eligible for prizes}.

不符合上述条件的参与者仍可参加挑战,但\textbf{没有资格获得奖品}。
\end{infobox}

\section{Team Rules / 团队规则}
\begin{itemize}
    \item Teams may consist of up to \textbf{3 members}. (团队最多可由 \textbf{3 名成员} 组成)
    \item Each team member must register individually. (每位团队成员必须单独注册)
    \item Teams can be mixed (students and professionals). (团队可以混合组成,即学生和专业人士)
\end{itemize}

\section{Prizes / 奖项}
\begin{enumerate}
    \item \textbf{Winner / 冠军}: \$5,000
    \item \textbf{1st Runner-up / 亚军}: \$3,000
    \item \textbf{2nd Runner-up / 季军}: \$2,000
\end{enumerate}

\section{Intellectual Property / 知识产权}
Participants retain full ownership of any intellectual property developed during the challenge. However, EY encourages open-sourcing the winning solutions to benefit the broader scientific community.

参赛者保留在挑战赛期间开发的任何知识产权的完全所有权。然而,EY 鼓励开源获奖解决方案,以造福更广泛的科学界。
