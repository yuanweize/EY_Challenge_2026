\section{General Questions / 一般问题}

\textbf{Q: What is the passing threshold? / 及格门槛是多少?}\\
A: You must achieve an $R^2$ score of at least \textbf{0.4} to receive a certificate of completion.

答:您必须达到至少 \textbf{0.4} 的 $R^2$ 分数才能获得结业证书。

\textbf{Q: Can I use other tools or languages? / 我可以使用其他工具或语言吗?}\\
A: Yes, you can use R, Julia, or other languages, but Python is highly recommended and fully supported with starter code.

答:是的,您可以使用 R、Julia 或其他语言,但强烈推荐使用 Python,并提供完整的入门代码支持。

\textbf{Q: Can I use external data? / 我可以使用外部数据吗?}\\
A: \textbf{Yes}, provided the data is free and publicly available to everyone. This ensures reproducibility. Examples include public weather/climate databases, soil maps, and elevation models.

答:\textbf{可以},前提是数据对所有人免费公开可用。这确保了可重复性。例如公共天气/气候数据库、土壤图和高程模型。

\textbf{Q: How are teams formed? / 团队如何组建?}\\
A: Teams can have up to 3 members. All members must register individually on the platform.

答:团队最多可由 3 名成员组成。所有成员必须在平台上单独注册。

\textbf{Q: Who owns the code? / 谁拥有代码的所有权?}\\
A: You (the participant) retain ownership of your intellectual property. However, sharing your solution with the community is encouraged after the competition.

答:您(参赛者)保留您的知识产权的所有权。然而,鼓励在比赛结束后与社区分享您的解决方案。
